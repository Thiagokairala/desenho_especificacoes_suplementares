
Documento de especificação suplementar é o responsável por englobar as características de um software que não estão inclusas no documento de caso de uso ou no documento de visão. Este documento, então, apresentará características não associadas às funcionalidades do software. 

Após a leitura deste documento, o leitor deverá ser capaz de compreender todo o funcionamento do sistema quando se pensa em Requisitos Não Funcionais. Requisitos estes que são representados pela sigla \textit{FURPS}.

\subsection{Escopo}

Este documento é referente à segunda \textit{sprint} da reconstrução do projeto \textit{Chamada Parlamentar 2}, e está relacionada aos casos de uso listados no documento de caso de uso gerado na referida \textit{sprint}.

\subsection{Visão Geral}

O documento a seguir está organizado em cinco sessões representando cada um dos aspectos ao qual o documento irá tratar.

\subsubsection{Comportamento}

Nesta sessão serão tratados comportamentos que o sistema deve ter que não são exclusivos a um caso de uso.

\subsubsection{Usabilidade}

Esta sessão tratará de requisitos específicos a usabilidade do sistema, aplicando métricas para que os mesmos possam ser medidos e validados.

\subsubsection{Confiabilidade}

Esta sessão será responsavel por expor as qualidades referentes a segurança, resiliência e tempo entre falhas do sistema.

\subsubsection{Desempenho}

Nesta parte do documento serão tratados requisitos referentes ao desempenho do sistema, tanto em relação a tempo de execução quanto em relação a número de consultas por operação.

\subsubsection{Questões de dimensionamento}

Na última sessão do documento serão descritas as espectativas futuras de crescimento do projeto, assim como riscos e questões a serem observadas e cuidadas para evitar futuros problemas.