%[As características de desempenho devem ser resumidas nesta seção. Inclua tempos de resposta específicos. Quando aplicável, faça referência a nomes de casos de uso de negócios relacionados.

%·         Tempo de resposta de uma transação (médio, máximo)

%·         Taxa de transferência (por exemplo, transações por segundo)

%·         Capacidade (por exemplo, o número de clientes ou transações que pode ser absorvido pelo negócio)

%·         Utilização de recursos: número de funcionários, capacidade de memória dos sistemas etc.]
O desempenho de um software pode ser medido por diversas métricas, para o projeto Chamada Parlamentar 2 serão utilizadas as métricas de tempo de resposta de uma transação e capacidade de acessos simultaneos.

\begin{itemize}

	\item{Tempo de resposta de uma transação}

		Nesta parte iremos separar o tempo de resposta em tempo médio e tempo máximo, para que assim possamos ter um referencial mais bem definido na hora de fazer a verificação e validação do sistema.

		O tempo de resposta médio entre as transações do sistema deverá ser de 0.5 segundo, enquanto o tempo máximo para uma transação nao deve ser superior a 1 segundo.

		Tendo em vista que as transações que envolvem o \textit{web service} da camara não estão sobre o controle da equipe de desenvolvimento do projeto, as transações que incluem acesso direto ao \textit{web service} não possuem tempo médio ou máximo de reposta.

	\item{Número de acessos simultaneos}

		O sistema deve ser capaz de suportar inicialmente 1000 acessos simultâneos, valor este que após os testes de estress ocorrerem pode ser aumentado.

\end{itemize}