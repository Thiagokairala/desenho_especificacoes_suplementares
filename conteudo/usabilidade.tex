
Usabilidade de um sistema de software se refere a facilidade na qual um usuário irá interagir com o mesmo. Esta facilitade é o que garante a eficiência no uso do sistema por um usuário leigo, ou seja, que nunca utilizou o sistema \textit{Chamada Parlamentar}. Existem muitas formas de maximizar a Usabilidade do sistema, tais como apresentar opção de treinamento, opção de ajuda, disposição dos itens de forma a facilitar o reconhecimento do usuário e etc.

Para o projeto \textit{Chamada Parlamentar 2} serão usadas as métricas de necessidade de treinamento, acessibilidade e quantidade de cliques.

\begin{itemize}

	\item{Necessidade de treinamento}

		O sistema será desenvolvido para que nenhum usuário necessite de treinamento apra utilizá-lo, tendo todas as suas funcionalidades em fácil acesso e auto-explicativas, para que assim o usuário se sinta confortável em utilizar o software numa frequência regular.

	\item{Acessibilidade}

		O sistema deverá seguir as práticas de acessibilidade presentes no \href{http://www.governoeletronico.gov.br/acoes-e-projetos/e-MAG}{Modelo de Acessibilidade em Governo Eletrônico}\cite{eMAG}.

	\item{Quantidade de cliques}

		Para que o usuário possa utilizar qualquer uma das funcionalidades do sistema, não deve ser necessário utilizar mais que 6 (seis) cliques a partir da tela inicial.

\end{itemize}