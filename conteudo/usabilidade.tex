Usabilidade de um sistema de software tem a ver com a facilidade na qual um usuário irá interagir com o mesmo, se precisará de um treinamento, quantos cliques são necessários para acessar cada uma das funcionalidades do sistema entre outras métricas.

Para o projeto Chamada Parlamentar serão usadas as métricas de necessidade de treinamento, acessibilidade e quantidade de cliques.

\begin{itemize}

	\item{Necessidade de treinamento}

		O sistema será desenvolvido para que nenhum usuário necessite ser treinado apra utilizá-lo, tendo todas as suas funcionalidades em fácil acesso e auto-explicativas para que assim o usuário se sinta confortável em utilizar o software numa frequência regular.

	\item{Acessibilidade}

		O sistema deverá seguir as práticas de acessibilidade presentes no \href{http://www.governoeletronico.gov.br/acoes-e-projetos/e-MAG}{Modelo de Acessibilidade em Governo Eletrônico}.

	\item{Quantidade de cliques}

		Para que o usuário possa utilizar qualquer uma das funcionalidades do sistema, não deve ser necessário utilizar mais que seis cliques a partir da tela inicial.

\end{itemize}