A confiabilidade de um software pode ser medida desde por quanto tempo o mesmo ficará disponível até a veracidade de suas informações.

Para o projeto Chamada Parlamentar 2 serão utilizadas as métricas de tempo de disponibilidade, precisão da informação, tempo entre as atualizações das informações do banco de dados e precisão matemática das contas estatísticas.

\begin{itemize} 

	\item{Tempo de disponibilidade}

		O sistema deve estar disponível por no mínimo 95,00\% do tempo gerando assim a confiança no usuário de que sempre poderá contar com a ferramenta.

	\item{Precisão da informação}

		A informação é garantida que seja verdadeira já que será retirada do \textit{web service} da camara dos deputados disponível pela lei dos dados abertos.

	\item{Tempo entre as atualizações das informações}

		As informações no banco de dados serão atualizadas diariamente durante a madrugada para evitar problemas de usuários tentando acessar o sistema e o mesmo apresentar informações incompletas.

	\item{Precisão matemática}

		O sistema garantirá aos usuários uma precisão matemática de duas casas decimais, e serão utilizados métodos para calcular comprovados por estudos matemáticos.

		Infelizmente algumas informações de alguns parlamentares vem quebrada impossibilitando assim a realização das contas estatísticas do mesmo, assim sempre que houver um caso deste tipo o usuário será avisado qual deputado está fora de análise pelo sistema.

\end{itemize}