
A confiabilidade de um software pode ser medida desde por quanto tempo o mesmo ficará disponível até a veracidade de suas informações. Outra forma de observar a confiabilidade do sistema é medindo o espaço de tempo entre um \textit{bug} e outro durante o funcionamento do sistema.

Para o projeto \textit{Chamada Parlamentar 2} serão utilizadas as métricas de tempo de disponibilidade, precisão da informação, tempo entre as atualizações das informações do banco de dados e precisão matemática das contas estatísticas.

\begin{itemize} 

	\item{Tempo de disponibilidade}

		O sistema deve estar disponível por no mínimo 95,00\% do tempo gerando assim a confiança no usuário de que sempre poderá contar com a ferramenta.

	\item{Precisão da informação}

		A informação apresentada pelo sistema \textit{Chamada Parlamentar 2} possui um nível de precisão elevado, já que todos os dados utilizados na informação serão obtidos pelo \textit{WebService} da Câmara dos Deputados, disponível após a confirmação da lei dos dados abertos do Governo Federal.

	\item{Tempo entre as atualizações das informações}

		As informações no banco de dados serão atualizadas diariamente durante a madrugada para evitar problemas de usuários tentando acessar o sistema e o mesmo apresentar informações incompletas. Esta atualização ocorrerá todos os dias as 05:00 A.M..

	\item{Precisão matemática}

		O sistema garantirá aos usuários uma precisão matemática de duas casas decimais, e todos os métodos de cálculo utilizados durante a geração da informação são comprovados matematicamente.

		Infelizmente, algumas informações advindas do sistema de \textit{WebService} da Câmara dos Deputados possuem erros no padrão de dados, impossibilitando, assim a realização dos cálculos para obtenção das estatísticas do parlamentar ou partido. Dessa forma, sempre que houver um erro de padrão de dado do \textit{WebService}, uma mensagem explicando a situação será apresentada ao usuário de forma clara e simples.

\end{itemize}